\documentclass{article}
\usepackage{gensymb, amsmath, float, graphicx, epstopdf}
\restylefloat{table}
\usepackage[margin=0.75in]{geometry}
\begin{document}

\title{Lab Write-up 4: Shielded-Loop Resonators}
\author{Michael Shen}
\maketitle


\section{Finding the Resonant Frequency}

\subsection{Measured Data}

\begin{table}[H]
\centering
\begin{tabular}{|c|c|c|}
\hline
Loop & $f_0$ (MHz) & $\vert\Gamma^{\prime}_{in}\vert$ \\ \hline
5 cm & 89.209 & 0.237 \\ \hline
9 cm & 49.774 & 0.260 \\ \hline
\end{tabular}
\end{table}

\subsection{Questions}

\begin{enumerate}
	\item Since $C \approx C^{\prime}l$, the longer, 9 cm, loop should have a greater capacitance.
	\item Since $L = \mu r [ln \frac{32r}{d}-1.75]$ for shielded-loop resonators and $\mu$ and $d$ are constant between the loops, the loop with the larger radius, the 9 cm loop, should have a greater inductance.
	\item Since $f_0 = \dfrac{\omega_0}{2\pi}$ and $\omega_0 = \dfrac{1}{\sqrt{LC}}$, the loop with the smaller $LC$ product will have a higher resonant frequency. Since the 5cm loop should have both a smaller $C$ and $L$, it should have a higher resonant frequency. This is confirmed by our experimental results as the 5 cm loop has a resonant frequency of 89.209 MHz compared to a resonant frequency of 49.774 MHz for the 9 cm loop.
\end{enumerate}


\section{De-embedding the Feedline}

\subsection{Measured Data}
\begin{table}[H]
\centering
\begin{tabular}{|c|c|c|c|}
\hline
Loop & $\vert\Gamma^{\prime}_{in}\vert$ & $\vert\Gamma^{\prime}_{in}\vert$ at $f_0$ - 2MHz & $\vert\Gamma^{\prime}_{in}\vert$ at $f_0$ + 2MHz \\ \hline
5 cm & 0.9735\angle 109.9\degree & 0.9745\angle 123.5\degree & 0.9744\angle 96.61\degree \\ \hline
9 cm & 0.9712\angle 118.1\degree & 0.9745\angle 143.6\degree & 0.9772\angle 93.61\degree \\ \hline
\end{tabular}
\end{table}

\subsection{Analysis}
\begin{enumerate}
	\item $\beta l_{5cm}$: $180 - 2\beta l = 109.9\degree \Rightarrow \beta l = 35.95\degree = 0.63 rad$ \\
		  $\beta l_{9cm}$: $180 - 2\beta l = 118.1\degree \Rightarrow \beta l = 31.45\degree = 0.55 rad$ \\
	\item $\beta l_{5cm}$ at $f_0 - 2 MHz = \beta l\times\dfrac{f_0 - 2MHz}{f_0} = 0.62 rad$ \\
		  $\beta l_{5cm}$ at $f_0 + 2 MHz = \beta l\times\dfrac{f_0 + 2Mhz}{f_0} = 0.64 rad$ \\
		  $\beta l_{9cm}$ at $f_0 - 2 MHz = \beta l\times\dfrac{f_0 - 2MHz}{f_0} = 0.53 rad$ \\
		  $\beta l_{9cm}$ at $f_0 + 2 MHz = \beta l\times\dfrac{f_0 + 2Mhz}{f_0} = 0.57 rad$ \\ 
	\item $\beta_{5cm} = \dfrac{\omega_0\sqrt{\varepsilon_r}}{c} = \dfrac{2\pi\times89.209\times10^6\times\sqrt{2.2}}{3.0\times10^8} = 2.77$ \\
	$\beta_{9cm} = \dfrac{\omega_0\sqrt{\varepsilon_r}}{c} = \dfrac{2\pi\times49.774\times10^6\times\sqrt{2.2}}{3.0\times10^8} = 1.55$
	\item $l_{5cm} = 4 + \pi r = 4 + 5\pi = 19.7 cm$ \\
	      $l_{9cm} = 5 + \pi r = 5 + 9\pi = 33.3 cm$ \\
	\item $\beta l_{5cm} = 2.77\times19.7 = 54.6 rad$ \\
		  $\beta l_{9cm} = 1.55\times33.3 = 51.6 rad$ \\
\end{enumerate}

\subsection{Questions}
\begin{enumerate}
	\item The 5 cm loop has a feedline with larger electrical length. $\beta = \omega\sqrt{L'C'}$. Since $\omega_0 = \dfrac{1}{\sqrt{LC}}$ and $L'C' = \dfrac{LC}{l^2}$, $\beta = \dfrac{1}{l}$ and $\beta l = 1$. Therefore, electrical length is independent of feedline length, so it is possible that the 5 cm loop does indeed have a larger electrical length.
	\item Both reflection coefficients are high, but the 9 cm loop had the higher reflection coefficient. We should desire a large magnitude of input reflection coefficient so that the line can be matched. For a lossless resonator, $\vert S_{11} \vert = 1$.
\end{enumerate}


\section{Finding R, L, and C}

\subsection{Measured Data}
\begin{table}[H]
\centering
\begin{tabular}{|l|l|l|l|}
\hline
Loop & R & L & C \\ \hline
5 cm & 0.671 $\Omega$  & 0.383 H  & $9.96\times10^{-6}$ F \\ \hline
9 cm & 0.731 $\Omega$  & 0.457 H & $3.07\times10^{-5}$ F \\ \hline
\end{tabular}
\end{table}

\subsection{Analysis}
$X_{a5} = 26.86$ \\
$X_{a9} = 16.44$ \\
$X_{b5} = 44.53$ \\
$X_{b9} = 46.93$ \\
$Q_{5cm} = 292.25$ \\
$Q_{9cm} = 166.41$ \\
\begin{enumerate}
	\item $L_{5cm} = \mu r[ln\dfrac{32r}{d} - 1.75] = 1.83\times 10^-7 H$ \\
		  $L_{9cm} = \mu r[ln\dfrac{32r}{d} - 1.75] = 3.9\times 10^-7 F$
	\item $C_{5cm} = C'l = \dfrac{\sqrt{\varepsilon_R}}{cZ_0}l = \dfrac{\sqrt{2.2}}{3\times10^8\times50} 0.05\pi = 1.55\times10^{-11} F$ \\
	$C_{9cm} = C'l = \dfrac{\sqrt{\varepsilon_R}}{cZ_0}l = \dfrac{\sqrt{2.2}}{3\times10^8\times50} 0.09\pi = 2.80\times10^{-11} F$ \\
\end{enumerate}

\subsection{Questions}

\begin{enumerate}
	\item The 5cm loop had a higher Q factor.
	\item The theoretical values do not match up at all with my calculated values. I'm not entirely sure where I made the mistake; the data seems reasonable and the theoretical result is definitely much more reflective of actual values I would expect. It seems I must have made a mistake substituting numbers into the equations for L and C, however I do not see an error there either.
	\item At the resonant frequency, all impedance is resistive, while lower and higher impedances introduce phase shifts resulting from imaginary impedance components. Faraday's Law of Induction describes how a change in magnetic flux will generate a voltage in the other coil. The flux is generated by the AC current and the amount recovered can be maximized when at the resonant frequency. Specifically at higher frequencies, the resistance will increase due to the skin effect, further depleting energy transfer.
\end{enumerate}

\section{Summary}
From this lab, I got an introduction to how wireless power transfer can be achieved, specifically by determining the resonant frequency and measuring characteristics of the shielded-loop resonator.

\end{document}