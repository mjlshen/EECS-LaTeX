\documentclass{article}
\usepackage{gensymb, amsmath, float, graphicx, epstopdf}
\restylefloat{table}
\usepackage[margin=0.75in]{geometry}
\usepackage{multicol}
\begin{document}

\title{Lab Write-up 5: Coupled Resonators and Voltage Rectification}
\author{Michael Shen}
\maketitle


\section{Measuring the Mutual Inductance}

\subsection{Measured Data}
As measured in the previous lab, $\omega_{0_{9cm}} = 49.774$ MHz and $\omega_{0_{5cm}} = 89.209$ MHz

\begin{table}[H]
\centering
\begin{tabular}{|c|c|c|c|c|}
\hline
Distance (cm)& 9cm $\vert\Gamma '_{in}\vert_{\omega=\omega_0}$     
			 & 9cm $\vert S_{21}\vert$ (dB) at $\omega = \omega_0$
			 & 5cm $\vert\Gamma '_{in}\vert_{\omega=\omega_0}$     
			 & 5cm $\vert S_{21}\vert$ (dB) at $\omega = \omega_0$ \\ \hline
2   		 & $0.195\angle-72.83\degree$ & -0.199 & $0.032\angle150.4\degree$  & -1.42   \\ \hline
4   	 	 & $0.324\angle115.8\degree$  & -0.682 & $0.575\angle109.5\degree$  & -1.847  \\ \hline
6   	 	 & $0.604\angle114.3\degree$  & -2.52  & $0.847\angle109.0\degree$  & -5.954  \\ \hline
8		     & $0.755\angle113.4\degree$  & -4.211 & $0.925\angle109.1\degree$  & -10     \\ \hline
10  		 & $0.843\angle113.2\degree$  & -6.201 & $0.941\angle108.9\degree$  & -12.078 \\ \hline
12 			 & $0.901\angle112.5\degree$  & -8.832 & $0.954\angle108.8\degree$  & -15.471 \\ \hline
14 			 & $0.925\angle112.9\degree$  & -10.91 & $0.961\angle108.9\degree$  & -18.42  \\ \hline
16  		 & $0.938\angle112.2\degree$  & -12.72 & $0.965\angle108.8\degree$  & -20.475 \\ \hline
\end{tabular}
\end{table}

\begin{table}[h]
\centering
\begin{tabular}{|c|c|c|}
\hline
Distance (cm)	  & 9cm $\vert\Gamma '_{in}\vert_{\omega=\omega_0}$      
			 	  & 5cm $\vert\Gamma '_{in}\vert_{\omega=\omega_0}$ \\ \hline
5cm $(30\degree)$ & $0.646\angle115.4\degree$ & $0.802\angle109\degree$   \\ \hline
5cm $(60\degree)$ & $0.893\angle112.4\degree$ & $0.903\angle109.4\degree$ \\ \hline
\end{tabular}
\end{table}
\subsection{Analysis}

\begin{enumerate}
	\item Since $R_{in} = Z_0\dfrac{1 - \vert\Gamma '_{in}\vert}{1 + \vert\Gamma'_{in}\vert}$ at $\omega=\omega_0$,
	\begin{table}[H]
	\centering
		\begin{tabular}{|c|c|c|}
		\hline
		Distance (cm) & 9cm $R_{in}$ $(\Omega)$ & 5cm $R_{in}$ $(\Omega)$ \\ \hline
		2  & 33.682 & 46.899 \\ \hline
		4  & 25.529 & 13.492 \\ \hline
		6  & 12.344 & 4.142 \\ \hline
		8  & 6.980 & 1.948 \\ \hline
		10 & 4.259 & 1.520 \\ \hline
		12 & 2.604 & 1.177 \\ \hline
		14 & 1.948 & 0.994 \\ \hline
		16 & 1.600 & 0.891 \\ \hline
		\end{tabular}
	\end{table}
	\item From $R_{in\vert\omega=\omega_0} = R_1 + \dfrac{(\omega_0M)^2}{R_2 + R_L}$, $M = \dfrac{\sqrt{(R_{in\vert\omega=\omega_0} - R_1)(R_2 + R_L)}}{\omega_0}$. Then, using $R_{1_{9cm}} = R_{2_{9cm}} = 0.47 \Omega$, $R_{1_{5cm}} = R_{2_{5cm}} = 0.41 \Omega$, and $R_L = 50\Omega$, values for the mutual inductance are shown below.
	\begin{table}[H]
	\centering
		\begin{tabular}{|c|c|c|}
		\hline
		Distance (cm) & $M_{9cm}$ (nH)& $M_{5cm}$ (nH)\\ \hline
		2  			  & 822 		  & 543  \\ \hline
		4  			  & 714 		  & 288  \\ \hline
		6  			  & 492 		  & 154  \\ \hline
		8  			  & 364 		  & 98.7 \\ \hline
		10 			  & 278 		  & 83.9 \\ \hline
		12 			  & 208 		  & 69.7 \\ \hline
		14 		  	  & 174		   	  & 60.8 \\ \hline
		16 			  & 152 		  & 55.2 \\ \hline
		\end{tabular}
	\end{table}
	\item After calculating $R_{in}$ as detailed in 1.2.1, the mutual inductances for the misaligned loops was calculated as detailed in 1.2.3 and are shown below.
	
	\begin{table}[h]
	\centering
		\begin{tabular}{|c|c|c|}
		\hline
		Distance (cm)	  & $M_{9cm}$ (nH) & $M_{5cm}$ (nH) \\ \hline
		5cm $(30\degree)$ & 458 		   & 179   			\\ \hline
		5cm $(60\degree)$ & 219 		   & 116 		    \\ \hline
		\end{tabular}
	\end{table}

	\item $\eta = \dfrac{4R_L^2(\omega_0M)^2}{((R+R_L)^2+(\omega_0M)^2)^2}$ !!!11!!
\end{enumerate}

\subsection{Questions}
\begin{multicols}{2}
\setlength{\premulticols}{1pt}
\setlength{\postmulticols}{1pt}
\setlength{\multicolsep}{1pt}
\setlength{\columnsep}{2pt}
	
	\begin{figure}[H]
	\centering
   		\includegraphics[scale=0.66]{./Matlab/9cm.eps}
	\end{figure}
	\begin{figure}[H]
	\centering
		\includegraphics[scale=0.66]{./Matlab/5cm.eps}
	
	\end{figure}
\end{multicols}

\begin{enumerate}
	\item The general decreasing trend is reflected with my measured mutual inductance values, but the data values themselves begin at a higher value and do not decrease as low as the values reflected with the plot. This may be measurement error due to our loops not being perfectly axially-aligned (this was eye-balled). Furthermore, the equations used to generate the provided plots are for filamentary loops; the loops we used in the lab had a non-negligible cross-sectional width and thickness.

	\item The mutual inductance for the misaligned loops was lower than that for aligned loops at the same distance away because the magnetic energy is being directed outwards, perpendicular to the first loop. By offsetting the second loop at an angle, some of the magnetic energy from the first loop is missing the second loop, resulting in a lower mutual inductance.
\end{enumerate}


\section{Strong and Weak Coupling}

\subsection{Measured Data}
\begin{table}[H]
\centering
\begin{tabular}{|c|c|c|}
\hline
Loop & Experiment & Theory \\ \hline
9cm & 2cm & 2.5cm\\ \hline
5cm & 1.5cm & 1.3cm\\ \hline
\end{tabular}
\end{table}

\subsection{Analysis}

\begin{enumerate}
	\item Since $\omega M = R + R_L$, $M_{9cm} = 1.61\times10^{-7}$ and $M_{5cm} = 8.99\times10^{-8}$. According to Figures 5.3 and 5.4 in the lab manual, this means the critical coupling distances are approximately 0.025m and 0.013m for the 9cm and 5cm loops, respectively.
\end{enumerate}

\subsection{Questions}

\begin{enumerate}
	\item Since $\omega M = R + R_L$, an increase in the resistance of the shielded-loop resonators results in an increase in the mutual inductance; this leads to an increase in the critical coupling distance.
	\item Since $\eta = \dfrac{R_L^2}{(R+R_L)^2}$ under strong coupling, an increase in the resistance of the shielded-loop resonators results in a decrease in power transfer efficiency.
	\item As the loops are moved closer together, the odd and even mode frequencies move farther and farther apart. This means that under strong coupling the operating frequencies must be changed accordingly with the distance so that the wireless power transfer system is operating at the new odd or even mode frequencies.
	\item The theoretical and experimental coupling distances are very similar, considering that both had a visual estimation involved.
\end{enumerate}


\section{Measuring the Rectifier}

\subsection{Measured Data}

\begin{enumerate}
	\item Rectifier input return loss at $\omega = \omega_0$: -0.47 dB
	\item Rectifier input impedance at $\omega = \omega_0$: 10.07 - j122.6 $\Omega$
\end{enumerate}

\subsection{Questions}

\begin{enumerate}
	\item The power level of the VNA must be increased to measure the rectifier because the rectifier is unmatched, resulting in some reflected power. Furthermore, the diodes will convert the negative side of the signal to positive for the rest of the circuit to smooth out.
\end{enumerate}

\section{Summary}
Pls no.
\end{document}