\documentclass{article}
\usepackage{gensymb, amsmath, float, graphicx, epstopdf}
\restylefloat{table}
\usepackage[margin=0.75in]{geometry}
\usepackage{multicol}
\begin{document}

\title{Lab Write-up 6: Impedance Matching and Frequency Tuning}
\author{Michael Shen}
\date{December 15, 2014}
\maketitle


\section{Testing the Matching Networks}

\subsection{Measured Data}
As measured in the previous labs and confirmed in this lab, $\omega_{0_{9cm}} = 46.973$ MHz.

\begin{enumerate}
	\item Measured output impedance for loop matching network \#1: 14.40 - j20.23 $\Omega$
	\item Measured output impedance for loop matching network \#2: 15.26 - j20.21 $\Omega$
	\item Measured output impedance for rectifier matching network: 2.312 - j30.54 $\Omega$
	\item Input return loss of matched rectifier: -21.904 dB
	\item Input return loss of matched system at 10cm: -26.527 dB
\end{enumerate}

\begin{table}[H]
\centering
\begin{tabular}{|c|c|}
\hline
Distance (cm) & Input Return Loss (dB) \\ \hline
2             & -0.702                 \\ \hline
4             & -1.880                 \\ \hline
6             & -5.117                 \\ \hline
8             & -10.631                \\ \hline
10            & -26.527                \\ \hline
12            & -10.627                \\ \hline
14            & -6.917                 \\ \hline
16            & -3.712                 \\ \hline
\end{tabular}
\end{table}

\subsection{Questions}

\begin{enumerate}
	\item A length of cable should not be used between the matching networks and the loops or the rectifier and its matching network because the cable will introduce an added electrical length unaccounted for when designing the matching network, very likely causing a phase shift and causing power to be reflected. It does not matter if there is a cable between the matched rectifier and the matched loops because these all have an impedance of $50 \Omega$, so no power will be reflected regardless of the length of cable.

\end{enumerate}


\section{Frequency-Tuned System}

\subsection{Measured Data}
\begin{table}[H]
\centering
\begin{tabular}{|c|c|c|}
\hline
Distance (cm) & Odd Mode Frequency (MHz) & Even Mode Frequency (MHz) \\ \hline
2             & 40.501                   & 63.172                    \\ \hline
4             & 43.372                   & 55.549                    \\ \hline
6             & 45.286                   & 53.008                    \\ \hline
8             & 46.474                   & 51.358                    \\ \hline
10            & 47.893                   & 49.672                    \\ \hline
12            & 48.817                   & 48.817                    \\ \hline
14            & 48.817                   & 48.817                    \\ \hline
16            & 48.817                   & 48.817                    \\ \hline
\end{tabular}
\end{table}

\begin{table}[H]
\centering
\begin{tabular}{|c|c|c|}
\hline
Distance (cm) & Constant Frequency $\vert S_{21}\vert$ (dB) 
& Frequency-Tuned $\vert S_{21}\vert$ (dB) \\ \hline
2             & -7.767                          & -0.621                       \\ \hline
4             & -4.622                          & -0.558                       \\ \hline
6             & -2.630                          & -0.663                       \\ \hline
8             & -1.176                          & -0.733                       \\ \hline
10            & -0.593                          & -0.743                       \\ \hline
12            & -0.938                          & -0.938                       \\ \hline
14            & -1.904                          & -1.904                       \\ \hline
16            & -3.360                          & -3.360                       \\ \hline
\end{tabular}
\end{table}

\subsection{Analysis}

\begin{enumerate}
	\item Using Eq. 6.33, $R_L = 14.83\Omega$, $\omega_0 = 46.974$ MHz, $R = 0.47\Omega$, and the mutual inductance values from the previous lab, the power transfer efficiency, $\eta$, at $\omega_0$ is (MATLAB code attached):
	\begin{table}[H]
	\centering
		\begin{tabular}{|c|c|c|}
		\hline
		Distance (cm) & M (nH)& $\eta_{\omega_0}$  \\ \hline
		2             & 822   & 0.4408  \\ \hline
		4             & 714   & 0.5358  \\ \hline
		6             & 492   & 0.7962  \\ \hline
		8             & 364   & 0.9280  \\ \hline
		10            & 278   & 0.9163  \\ \hline
		12            & 208   & 0.7733  \\ \hline
		14            & 174   & 0.6491  \\ \hline
		16            & 152   & 0.5519  \\ \hline
		\end{tabular}
	\end{table}

	\item Using Eq. 6.39, $\eta_{odd} = \dfrac{R_L^2}{(R+R_L)^2} = 0.9395$ for distances 2, 4, 6, 8, and 10 cm. Distances from 12-16 cm were not strongly coupled and did not have an odd mode frequency.
	\item Using Eq. 6.37, the maximum power transfer efficiency values are plotted below.
	\begin{figure}[H]
		\centering
   		\includegraphics[scale = 0.85]{./Matlab/Analysis2_1.eps}
	\end{figure}
\end{enumerate}

\subsection{Questions}

\begin{enumerate}
	\item The input return loss is at a minimum at the matching distance and increases as the loops are moved away from it. This means that the input reflection coefficient of the matched system was also at a minimum at the matching distance and increased as the loops are moved away from it because the input return loss is effectively the magnitude of the input reflection coefficient.
	\item The theoretical and measured power transfer efficiencies follow the same general shape. However it appears that the measured data peaks at 8cm instead of the theoretical 10cm (MATLAB code attached).
	\begin{figure}[H]
		\centering
   		\includegraphics[scale = 0.85]{./Matlab/Question2_2.eps}
	\end{figure}
	\item Yes, the coupling distance of 10 cm has the minimum return loss and maximum $S_{21}$.
	\item For distances from 2-10cm, the system was tuned to the odd mode frequency. After 10cm, the system is no longer in strong coupling so $\omega_0$ is the frequency the system was tuned to (MATLAB code attached).
	\begin{figure}[H]
		\centering
   		\includegraphics[scale = 0.85]{./Matlab/Question2_3.eps}
	\end{figure}
	\item The critical coupling distance is different because the system is now frequency tuned, allowing power to be efficiently transferred across further distances, also increasing the critical coupling distance. The unmatched critical coupling distance for the 9cm loops was 2.5cm, compared to the matched critical coupling distance determined in the lab to be 10cm.
\end{enumerate}

\section{Summary}
In this lab I learned the importance of frequency tuning, specifically that in strong coupling, odd and even mode frequencies result in less power loss than the resonant frequency. Furthermore, I learned how to design matching networks and rectifiers to allow sufficient power transfer to light up an LED over much larger distances when compared to unmatched systems.

\end{document}